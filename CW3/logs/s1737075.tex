\documentclass[titlepage, 12pt]{extarticle}
\usepackage[margin=1in]{geometry}
\usepackage{fancyhdr}
\usepackage{lastpage}
\usepackage{graphicx}

\setlength{\parindent}{0pt}
\lhead{Software Engineering, Coursework 3}
\rhead{s1737075}
\pagestyle{fancy}
\begin{document}
\title{{\bf Inf2C Software Engineering \\ Coursework 3 Project Log}}
\author{Dylan Thinnes, s1737075}
\date{November 28, 2018}
\maketitle

\section{Overview}
This report will go over three aspects of every day in which work was done:
\begin{itemize}
    \item Plans for the day.
    \item Achieved results. This includes:
        \begin{itemize}
            \item Items of work done for that day
            \item The time spent for that day
        \end{itemize}
    \item Reflections on improvements and obstacles.
\end{itemize}

The majority of this log is modeled after my commit history and personal
recollection of events as I typed them up.

\section*{November 17th}
\subsection{Plans for the Day}
We began the day by looking at the coursework specification. Most importantly,
we sought to answer three main questions before actually doing any coding:

\begin{enumerate}
    \item {\bf What work is there to be done? }\\
        This was split into three categories:
        \begin{enumerate}
            \item Implementation of individual components\\
                Each component must be defined according mostly to the specification
                laid out in our class diagram. Appropriate types must be chosen for
                internal variables, the functions internal to the class must be
                written, and getters must be written for any private variables we wish
                to expose.
            \item Gluing components together according to user stories\\
                This work resides mostly in AuctionHouseImp. It must implement each
                documented user story and, in doing so, tie together the individual
                internal functions of each related class to produce the user story's
                primary outcome.
            \item Testing standalone components and relationships\\
                JUnit tests must finally be written to be sure that all of these user
                stories and the individual components behave correctly. This is
                especially important for ensuring functions abort appropriately when
                bad inputs are given.
        \end{enumerate}
    \item {\bf How will we divide the work to be done? }\\
        Generally, we resolved to split coding jobs as we went along. Mikey has more
        extensive experience with testing, so he would have a bias towards that.
        Simultaneously, I find general grunt coding less tedious, and would slightly
        bias myself towards that work.
    \item {\bf What tooling and external specification should we consider? }\\
        Our workflow would remain unchanged from our usual ones, using vim/Emacs, with
        git, latex, and bash utilities.

        We discussed the relative merits of switching over to a more accepted Netbeans
        or Eclipse-based workflow, but considered it tedious in comparison to writing
        our own scripts for common tasks and using our already powerful editors to
        write up everything.

        In the way of coding conventions, we read through the Oracle guidelines for
        coding styles. We found nothing we disagreed with, and resolved to follow it.

        We read some brief blog posts about JUnit, and the documentation of logger, to
        be sure we could troubleshoot as necessary and use them effectively.
\end{enumerate}

After this, we decided to implement the skeletons of our structure, and to make
workflow tweaks as necessary since this was the earliest stage.\\

\subsection{Achieved Results}
I continued the day by developing skeletons and proper class hierarchies of all
of our components. This way, we have everything clearly named, preventing
naming conflicts in our function and variable definitions between merges.\\

I created a `compile` script, to compile the code before running it. This
made typechecking very simple and immediate, so we knew that we had an error
the second we typed it.\\

I modified our test script to pipe stderr to stdout - this would let us run
grep, tail, and other shell utilities on our tests.\\

\subsection{Reflection}
Workflow was significantly improved so that we can identify type errors and
failed tests quickly. This will be especially useful in the long run.\\

Also, most major skeletons were set up successfully, so we are prepared to
begin our implementation with the confidence that we will not have to resolve
variable naming conflicts, as everything has been determined here ahead of
time.

{\bf Hours Spent: 7}

\section{November 18th}
\subsection{Plans for the Day}
\begin{itemize}
    \item Refine our class inheritance structure, especially for actors
    \item Ensure correct encapsulation
    \item Write a user story in AuctionHouseImp
\end{itemize}

\subsection{Achieved Results}
\subsubsection{Refine class inheritance structure}
After brief discussion, I reworked the hierarchy of our actors (Auctioneer,
MemberOfPublic, Seller, Buyer).\\

Previously, all actors implemented the Actor interface, which required the
"viewCatalogue" function. Instead, the Actor became a class unto itself and
implemented viewCatalogue for all actors.\\

Some discussion was had about the inclusion of MemberOfPublic. Mikey argued
that because it wasn't in the spec, we shouldn't include it. I argued that we
should include it since it would serve as an example of the Actor class's
usability as an anonymous, read-only actor. Mikey relented on the grounds that
it wouldn't complicate tests or user stories.\\

\subsubsection{Ensure correct encapsulation}
In the way of ensuring correct encapsulation, I found many skeleton
implementations where I had mistakenly allowed class attributes to be public
where they needed to be private. I privatized those variables, and wrote some
getters.

\subsubsection{Implement a user story}
I began work on the lot closing transaction system. Mikey also made progress on
this - upon merging in his changes, I found that our code is very pleasingly
similar.\\

In this endeavour, I also made sure to verify an auctioneer as existent for
every lot opening/closing event.

\subsection{Reflection}
Our work from yesterday was very useful in quick iteration of the code design -
typing let us catch most if not all bugs early.\\

The tests e were provided, and some of Mikey's beginning work with tests, was
very useful for catching other bugs in the lot closing system. I think it is
advantageous to have tests written ahead of time as code mutates.

{\bf Hours Spent: 6 }

\section{November 23rd}
\subsection{Plans for the Day}
\begin{itemize}
    \item Track and fix bugs
        Enough logic has been written at this early point that it is important
        that we go over it to find potential runtime issues.
\end{itemize}

\subsection{Achieved Results}
I found a few bugs, especially in the payment transfer logic, that needed to be
immediately fixed. Other logic in the way of verifying transfers and reacting
to payment errors could be added in.

\subsection{Reflection}
It quickly became apparent that our modular design and our choice of workflows
was an excellent choice for this project. Whenever we were writing code in the
same parts of the codebase, merges would usually be automatic, thanks to git's
merge system.\\

Also, the use of TODOs across the system did not introduce any fragmentation of
priorities or knowledge - thanks to our environment, it was simple to write a
quick grep command that would find any active TODOs in the system.\\

We clearly work well as a team. We easily transfer between parts of the
codebase, can communicate issues quickly, and are comfortable with one
another's code. This lets us extend and review one another's work rapidly.

{\bf Hours Spent: 3 }
\section{November 24th}
\subsection{Plans for the Day}
\begin{itemize}
    \item Work together on report, esp. class diagram
    \item Catch up on latex transcription of my project log
\end{itemize}

\subsection{Achieved Results}
The report was finished piecemeal --- initially we discussed the differences
between our CW2 report and what we had actually implemented in CW3. Once we
were clear on that subject, Mikey wrote up a skeleton set of titles. Then, he
fleshed out paragraphs for the first two of three sections. We refrained from
writing too much more --- we wanted feedback first.\\

I also began transcribing what had up until then been a project log only in
plaintext into latex.\\

\subsection{Reflection}
The deadline is fast approaching, and much of our implementation, especially on
a macro level, is complete. So, it is both logical and important to begin our
report.\\

Everything we wanted to achieve was achieved.\\

{\bf Hours Spent: 3 }
\section{November 25th}
\subsection{Plans for the Day}
Planned to finish off my side of reports.

\subsection{Achieved Results}
I edited Mikey's previous project report paragraphs and wrote up sections Mikey
hadn't touched. Mostly, this included documenting structural decisions and
improving explanations of differences between CW2 specification and CW3
implementation.

\subsection{Reflection}
Mikey and I have different writing styles --- generally, he is more concise. I
find it useful to run my sentences past him before committing them to git.

{\bf Hours Spent: 2 }
\section{November 26th}
\subsection{Plans for the Day}
Just look over the code, make sure that nothing is amiss.

\subsection{Achieved Results}
Main issue I could find was authorship attributions. I changed all instances of
"@author djt" to be "@author Dylan Thinnes, Michael Andrejczuk" instead.

\subsection{Reflection}
Fewer issues are coming to the forefront than before. It is really a question
of getting smaller details tidied away before the coursework deadline.\\

The lack of feedback at this point is a problem --- we will have to wait until
feedback comes out to make our final documentation and implementation edits.

{\bf Hours Spent: 1 }

\section{November 28th}
\subsection{Plans for the Day}
\begin{itemize}
    \item Finish off all documentation in lieu of looming deadline
    \item Finish transcribing, compiling, and submitting my project log
\end{itemize}

\subsection{Achieved Results}
Today, I made no code changes.\\

All efforts were directed towards taking into account the feedback we got, and amending our class diagram appropriately.\\

I also finished transcribing my log, and final submissions were made. This took longer than I expected.

\subsection{Reflection}
This entire coursework was a useful exercise in reading a design prompt, making
appropriate plans and specifications, and then seeing how those specifications
can be well realized in a well-tested implementation.\\

Keeping a log, and as much documentation as we did, also put into scope how
difficult the issue of documentation must be for real---life, large projects.\\

Finally, managing these issues a team, with a workflow and multiple editor
environments, served as a useful lesson in mutual team dynamics.

{\bf Hours Spent: 3 }

\end{document}

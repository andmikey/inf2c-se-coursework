\documentclass[titlepage, 12pt]{extarticle}
\usepackage[margin=1in]{geometry}
\usepackage{tikz}
\usepackage{../../CW2/tikz-uml}
\usepackage{fancyhdr}
\usepackage{lastpage}
\usepackage{pdfpages}
\usepackage{verbatim}

\lhead{Inf2C: SE Coursework 3}
\rhead{s1703773 \& s1737075}
\cfoot{\thepage~/~\pageref{LastPage}}
\pagestyle{fancy}
\begin{document}
\title{{\bf Inf2C: Software Engineering \\Coursework 3 \vspace{2em}\\ Creating an abstract implementation of a auction house system}}
\author{
\begin{tabular}{l  c}
  Michael Andrejczuk & s1703773 \\
  Dylan Joseph Thinnes & s1737075
\end{tabular}
}
\date{November 27, 2018}
\maketitle
\tableofcontents

\section{UML class diagram}
Due to its size, this is provided as the last page of the document.

\noindent Note that for brevity we leave off a number of associations:
\begin{itemize}
\item Where Buyers / Auctioneers have a list of Lots associated with them, we do not show this
\item The CatalogueEntry to AuctionHouseImp relation is ordered
\item The addressBook map in AuctionHouseImp is not shown
\end{itemize}

\section{High-level design description}
We give an overview on a per-class basis, giving particular attention to
deviations from our Coursework 2 specification.

\subsection{Actors}
\begin{itemize}
    \item {\bf Actor} In our CW2, this was an interface which supported
        receiving/sending messages via the MessageService. We now implement it
        as a superclass, abstracting the concept of an actor associated with
        the system. In particular, an actor has two attributes associated with
        them:
        \begin{enumerate}
            \item address (String), the messaging address. By including this in
                the superclass we ensure all actors have consistent behaviour in
                how their address is set and retrieved.
            \item auctionhouse (AuctionHouse), the AuctionHouse associated with
              the actor. This allows us to reference back to the auctionhouse object if needed,
              keeping our system understandable. It also means we could easily extend the system to
              support multiple Auction Houses. 
        \end{enumerate}
        The Actor also now has the additional viewCatalogue function, which was
        previously implemented by all actor inheritors separately. It serves
        merely as an internal convenience function to calling viewCatalogue on
        the instance of AuctionHouseImp.
    \item {\bf RegisteredUser} This now inherits from Actor, and extends it by
        requiring a specific username for each instance. This separates
        registered actors, such as Auctioneer, Client, and Seller, who need a
        username, from anonymous actors, such as MemberOfPublic.
    \item {\bf Client} The client retains some functionality in Coursework 2,
        since it provides banking credentials. However, it is no longer responsible
        for handling personal details, as personal details are better put in
        RegisteredUser. Seller and Buyer inherit from Client, ensuring that both
        have relevant banking authentication details.
    \item {\bf Seller} This remains the same as in CW2, except that all methods
        are implemented in superclasses.
    \item {\bf Buyer} This remains the same as in CW2, except that all methods
        are implemented in superclasses. 
    \item {\bf Auctioneer} This remains largely the same as in CW2, except the
        Auctioneer now tracks if it is currently administering a lot. This needs
        to be tracked in order to ensure an auctioneer cannot run more than one
        auction at once. By keeping this in the Auctioneer class rather than
        storing this information in the AuctionHouse we ensure high cohesion
        and low coupling.
\end{itemize}
\subsection{Auction House}
AuctionHouseImp has largely the same shape as
our CW2 implementation of AuctionHouse, with three major exceptions.
        \begin{enumerate}
            \item {\bf Lots} These are now indexed by a Map of lotIds, which are
                Integers, to Lots. Since all lots are indexed using these, it is
                more efficient and clearer to have the data structure imply how
                we find and use these lots.
            \item {\bf Parameters} Parameters as provided during initialisation
                are stored in AuctionHouse as an attribute of type Parameters. We
                considered copying the attributes directly into AuctionHouse,
                but decided against it as it would tightly couple AuctionHouse
                to changes in the implementation of Parameters.
            \item {\bf Auctioneers} Our previous implementation did not store
                auctioneers, which was a major oversight. Now, we store them as
                a list, which simplifies interacting with and accessing them for
                purposes of lot administration.
        \end{enumerate}
\subsection{Lot and Associates}
\begin{itemize}
    \item {\bf CatalogueEntry} This remains the same as in CW2.
    \item {\bf Lot} This remains the same as in CW2, except that there are no
        longer LotInformation and LotDescription attributes. 
    \item {\bf LotStatus} This follows our implementation in CW2, with the
        exception that the ``Pre-auction'' status has been replaced with
        ``Unsold''. This allows us to auction a lot again if it did not sell
        previously, a detail we had omitted in our CW2. 
    \item {\bf Bid} This remains the same as in CW2.
\end{itemize}

\section{Implementation Decisions}
We note particular points of interest in our implementation decisions:
\begin{itemize}
    \item {\bf Abstraction of actors}: we chose to represent all classes that may represent an individual (Seller, Buyer, Auctioneer, MemberOfPublic) as subclasses of Actor, RegisteredUser, and Client. This allowed us to follow a strongly object-oriented approach in which we inherit methods and necessary attributes, guaranteeing an easily composed and incrementally growing set of user features as we deal further down the inheritance chain, such as messaging on the first level (Actor), usernames on the second level (RegisteredUser), and banking authentication on the third (Client).
    \item {\bf Removal of Client methods}: in our CW2, each Client had some set
        of actions they performed, represented by a method: for example,
        `bidOnLot' for Buyer. These have all been moved into AuctionHouseImp, and actors can interact with the AuctionHouse as necessary to mutate the program state. This models the real world more closely - most dealings are by actors with the auction house, not between each other. It also has the far more important benefit of keeping logic properly encapsulated to each specific actor instance, whether that is Seller or Buyer or other actors. This keeps our system less dependent on the interfaces of these actors, and gives us more freedom for improvement in the future.
    \item {\bf Consistent choice of collection classes}: we use ArrayLists and
        HashMaps where needed. Their APIs are well understood by even the most basic Java programmers, making our system easy to understand and, by extension, maintain and extend.
\end{itemize}

\includepdf[pages=1,fitpaper]{class_diagram.pdf}

\end{document}

\documentclass{article}

\begin{document}
\title{Inf2C: Software Engineering \\Coursework 1}
\author{Michael Andrejczuk: s1703773 \\
  Dylan Joseph Thinnes: s1737075}
\date{19th October, 2018}
\maketitle

\section{Introduction}

\section{Stakeholders}
We can consider the following stakeholders:
\begin{itemize}
\item {\it Auction House}: the Auction House wants to replace its existing bidding system with a software that will
    \begin{itemize}
    \item Perform existing in-house functions that the auction house currently manages manually. such taking in new bids.
    \item Open up new features to their clients (bidders, sellers, and the public) so that lots can be viewed, edited, and bid on through an online interface and without being physically present.
    \end{itemize}
\item Auctioneer: Manages auctions and oversees the bidding process by taking in new bids. They want to be aware of bids as they come in through the platform, and want to notify those involved in the bidding process as things change.
\item Buyer: Watches and bids on lots under auction. They want to see lots and their associated images, descriptions, and valuations, and want to mark interest for and bid on the lots that they want.
\item Seller: Brings lots to the auction and lists their starting prices and reserve prices. They want to be able to manage their lots and their images, descriptions, and valuations. They want to be notified about bids on their lots and told of the final prices (hammer and hammer minus seller's commision) of any lots they have sent up for auction.
\item Member of the Public: Wants to peruse the existing lots. A member of the public without an existing account needs access to seeing lots and possibly their valuations without knowing more important, buyer/seller-only details.
\item Expert: Helps to evaluate a given lot and guide the process of setting up a lot, especially its valuations. They need access similar to that of a Member of the Public, but with additional abilities to contact the seller and negotiate aspects of a lot.
\item Lot Display Professional: For lack of a better term, these are the professionals involved in taking photos and writing descriptions of a lot. They are interested in the multimedia capabilities of a lot's listing, as that will affect their process for marketing it and making attractive to potential bidders.
\item App Provider: The group responsible for maintenance of the hardware and software involved in distributing the app and keeping the app running. These may be in-house professionals for the Auction House or a another contracted agency. They are interested in the simplicity of maintaining and distributing the software and handling its interconnection between other stakeholders.
\end{itemize}
\section{System State}
The system consists of five major state components:
\begin{itemize}
\item Auctions: A list of auctions is kept which each have a list of associated Lots and a list of associated Auctioneers, as well as a minimum bid increment. Each auction also has a start date and time. If the auction is in progress, it has whatever item is currently on auction.
\item Auctioneers: Auctioneers administrate the process of an auction. They keep a list of all auctions they will administrate.
\item Lots: Lots have an associated seller that sells them and an auction at which they will be auctioned off. A lot also has a high/low estimate, a description, and images. If the lot has multiple items, each item may come with its own description and images. A reserve price (which will only be visible to the seller) and a list of interested bidders, which have marked their interest before the lot has gone on auction. If the lot is actively on auction or has now been auctioned off, a list of bids is kept along with the final, winning bid.
\item Buyers: Buyers have personal information, contact details, and bank information. They also have a list of lots which they have marked their interest in. If a lot they have subscribed to is active, they store all bids they have made on the lot with special reference to the last bid they made.
\item Sellers: Sellers, much like Buyers, have personal information, contact details, and bank information attached. They also keep a list of lots that they currently have registered with the Auction House.
\end{itemize}
\section{Use Cases}

\section{Use case diagram}

\section{Non-Functional Fequirements}

\section{Ambiguities, Subtleties, Incompleteness}

\end{document}

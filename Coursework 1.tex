\documentclass[titlepage]{article}

\begin{document}
\title{{\bf Inf2C: Software Engineering \\Coursework 1 \vspace{2em}\\ Capturing requirements for an auction house system}}
\author{
\begin{tabular}{l  c}
  Michael Andrejczuk & s1703773 \\
  Dylan Joseph Thinnes & s1737075
\end{tabular}
}
\date{October 19, 2018}
\maketitle

\section{Introduction}

\section{Stakeholders}
We can consider the following stakeholders:
\begin{itemize}
\item {\it Auction House}: the Auction House wants to replace its existing bidding system with a software that will
    \begin{itemize}
    \item Perform existing in-house functions that the auction house currently manages manually. such taking in new bids.
    \item Open up new features to their clients (bidders, sellers, and the public) so that lots can be viewed, edited, and bid on through an online interface and without being physically present.
    \end{itemize}
\item {\it Auctioneer}: Manages auctions and oversees the bidding process by taking in new bids. They want to be aware of bids as they come in through the platform, and want to notify those involved in the bidding process as things change.
\item {\it Buyer}: Watches and bids on lots under auction. They want to see lots and their associated images, descriptions, and valuations, and want to mark interest for and bid on the lots that they want.
\item {\it Seller}: Brings lots to the auction and lists their starting prices and reserve prices. They want to be able to manage their lots and their images, descriptions, and valuations. They want to be notified about bids on their lots and told of the final prices (hammer and hammer minus seller's commision) of any lots they have sent up for auction.
\item {\it Member of the Public}: Wants to peruse the existing lots. A member of the public without an existing account needs access to seeing lots and possibly their valuations without knowing more important, buyer/seller-only details.
\item {\it Expert}: Helps to evaluate a given lot and guide the process of setting up a lot, especially its valuations. They need access similar to that of a Member of the Public, but with additional abilities to contact the seller and negotiate aspects of a lot.
\item {\it Lot Display Professional}: For lack of a better term, these are the professionals involved in taking photos and writing descriptions of a lot. They are interested in the multimedia capabilities of a lot's listing, as that will affect their process for marketing it and making attractive to potential bidders.
\item {\it App Provider}: The group responsible for maintenance of the hardware and software involved in distributing the app and keeping the app running. These may be in-house professionals for the Auction House or a another contracted agency. They are interested in the simplicity of maintaining and distributing the software and handling its interconnection between other stakeholders.
\end{itemize}
\section{System State}
The system consists of five major state components:
\begin{itemize}
\item {\it Auctions}: A list of auctions is kept which each have a list of associated Lots and a list of associated Auctioneers, as well as a minimum bid increment. Each auction also has a start date and time. If the auction is in progress, it has whatever item is currently on auction.
\item {\it Auctioneers}: Auctioneers administrate the process of an auction. They keep a list of all auctions they will administrate.
\item {\it Lots}: Lots have an associated seller that sells them and an auction at which they will be auctioned off. A lot also has a high/low estimate, a description, and images. If the lot has multiple items, each item may come with its own description and images. A reserve price (which will only be visible to the seller) and a list of interested bidders, which have marked their interest before the lot has gone on auction. If the lot is actively on auction or has now been auctioned off, a list of bids is kept along with the final, winning bid.
\item {\it Buyers}: Buyers have personal information, contact details, and bank information. They also have a list of lots which they have marked their interest in. If a lot they have subscribed to is active, they store all bids they have made on the lot with special reference to the last bid they made.
\item {\it Sellers}: Sellers, much like Buyers, have personal information, contact details, and bank information attached. They also keep a list of lots that they currently have registered with the Auction House.
\end{itemize}
\section{Use Cases}

\section{Use case diagram}

\section{Non-Functional Requirements}
Non-Functional Requirements are categorized into five sections:
\subsection{Security}
\begin{itemize}
\item {\it Bidding Authentication}: The auction process for a lot, and placing bids for a lot, must be resticted by users that have proper authentication as buyers and that have shown interest in the lot before auction. Therefore, the authentication system must be well secured to prevent people who do not meet these requirements from having access. There must be no possibility of a user elevating their access without explicit permission from the auction house.
\item {\it Bank Transactions} Bank transactions, since they deal with money, are especially sensitive to tampering. Such transactions, as well as the personal data required to make them, must be encrypted and verified beforehand to make sure both eavesdropping and untoward manipulation of the transaction are impossible.
\item {\it Anonymity} A bidder may want to keep their identity secret in high-profile contexts. In this case, the personal data of a user must be secured so that authentication must be duly provided to access this personal data. Furthermore, a log must exist so that administratie changes can be correctly audited and reverted in the unlikely event of a breach.
\end{itemize}
\subsection{Usability}
\begin{itemize}
\item {\it Internationalisation Support} Bidders may hail from a variety of countries and backgrounds, and to capture the largest number of potential bidders an auction house must target them through support for their language. This extends to currencies especially - when dealing with large bids, different currencies may be used to pay a single sum. Conversion with these currencies must be exact.
\item {\it Simple Interface} Since the bidding process is fast-moving and high-value, customers cannot be encumbered by the interface in making their decisions. The interface must be streamlined towards the fundamental functions of auctioning, such as bidding, setting a hammer price, or listing a lot.
\item {\it Multi-Platform Support} Clients will use a variety of platforms to access the auction house, as will members of the public. Clients that are accessing the platform from afar especially may be on the go and need mobile platform support.
\item {\it Notification Customizability} Information overload is a serious concern in a fast-moving context such as an active bidding session. Customers must be given only the notifications they need in order to carry out their bids, and auctioneers must only receive the notifications necessary to administrate an auction.
\end{itemize}
\subsection{Performance}
\begin{itemize}
\item {\it Scalability} As auctions may happen simultaneously, as the auction house may expand to multiple venues, and as the auction house attracts more clients, the system must not become slow or sluggish. The ability to scale but up and down as needed by the business is critical.
\item {\it Low-Latency / Low-Bandwidth} Customers in transit or from afar will be placing bids. Since the placing of a bid can represent major investment, and the acceptance of a bid affects the outcome of an auction seriously, all bids must be delivered quickly and even over patchy connections. This requires many of the transactions to have little network overhead.
\item {\it Low-Cost Computer Support} Similar to Multi-Platform Support under the Usability subsection, Clients may be connecting through limited PCs or underpowered mobile devices. The interface to bid must not slow down on such devices. This may involve choosing investing development time into platform-specific optimizations and fixes.
\end{itemize}
\subsection{Reliability}
\begin{itemize}
\item {\it Concurrency Management (ACID)} Transactions, especially bids, will come in asynchronously but must be processed sychronously and durably. Using a database that follows the ACID principles would be strong first step towards making sure that bids processed in the correct order and not made to overlap or interfere with one another.
\item {\it Integration Testing} Failure in the middle of an auction is not an option. All user scenarios and actions that a user can take would best be adequately tested using some automated testing framework.
\end{itemize}
\subsection{Maintainability}
\begin{itemize}
\item {\it Conceptual Modularization} In order to speed up training, understanding, and diagnosing the system, the program must be properly split into easily-digested modules that relate to a specific concept or process that is needed. This way, whenever an error occurs, it is self-evident what it causes by where it comes from. Whenever a new feature is needed, it is self-evident where the new feature should go and feature creep can be avoided.
\item {\it Unit Testing} Often, valuable hours are wasted on chasing errors due to regressions or diagnosing an in-production problem. The hot-fixing of production problems may also gradually introduce more and more cruft into a code-base, making ithard to predict and maintain further down the road. Unit testing is ideal for catching bugs early and keeping regressions out of the code-base on every change.
\end{itemize}

\section{Ambiguities, Subtleties, Incompleteness}

\end{document}

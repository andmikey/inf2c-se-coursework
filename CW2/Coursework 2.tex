\documentclass[titlepage, 12pt]{extarticle}
\usepackage[margin=1in]{geometry}
\usepackage{tikz}
\usepackage{tikz-uml}
\usepackage{fancyhdr}
\usepackage{lastpage}
\usepackage{pdfpages}
\usepackage{verbatim}

\lhead{Inf2C: SE Coursework 2}
\rhead{s1703773 \& s1737075}
\cfoot{\thepage~/~\pageref{LastPage}}
\pagestyle{fancy}
\begin{document}
\title{{\bf Inf2C: Software Engineering \\Coursework 2 \vspace{2em}\\ Creating a software design for an auction house system}}
\author{
\begin{tabular}{l  c}
  Michael Andrejczuk & s1703773 \\
  Dylan Joseph Thinnes & s1737075
\end{tabular}
}
\date{November 5, 2018}
\maketitle

\tableofcontents
\newpage

\section{Introduction}
We provide a software design for an auction house system, codenamed {\it Auctionista}. Our design is consistent with the requirements design undertaken in Coursework 1. We provide UML class models for the key actors involved in the system, and sequence diagrams and behavior descriptions for use-cases identified in Coursework 1. For further details, please refer to the specification for Coursework 2.

\section{Static model}
\subsection{UML class model}
Due to its size, this is provided as the last page of the document.

\subsection{High-level description}
\noindent Firstly, an overview on a per-concept basis:
\begin{itemize}
\item {\bf Actor: } The actors in the system --- Seller, Buyer, Auctioneer, MemberOfPublic --- are represented as implementations of the Actor interface. This means that:
  \begin{itemize}
  \item By separating interface and implementation, we follow good design principles.
  \item The Actor class ensures a consistent interface for message exchange using the MessageService singleton.
  \item We can encapsulate the handling of messages of different kinds on a per-object basis, making it easy to maintain and add more message types as needed.
  \item As each implementation of Actor will have a different way of handling various receiveMessage() instances, we leave these up to the classes to implement, rather than specifying them in Actor. This ensures good cohesion. 
  \end{itemize}
\item {\bf MessagingService: } Messages are sent and received only for objects which implement the Actor interface. In the case of this system, all messages passed around the system will relate to a lot. This means that we only define methods for Actors to receive messages which pertain to a given Lot and some change in status (see point below). However, we can easily create new receiveMessage() functions for Actors which take different parameters. 
\item {\bf LotUpdateMessage: } This class specifies all the possible state-changes a lot may undertake. By providing these states as an enum, we ensure maintainability and modularisation: by adding a new possible state to the enum we update all the classes which receive messages pertaining to lots.
\item {\bf RegisteredUser: } This represents superclass the actors who have an account on the system, allowing us to separate them from members of the public, who we assume to not have accounts. Using inheritance, we ensure consistent behavior for these users. Currently, this is only ensuring that all have a username: however, this is easily extensible to also include, for example, passwords and two-factor-authentication.
\item {\bf Client: } A subclass of RegisteredUser, this represents Sellers and Buyers. These two actors use the system in a similar way, so we ensure a similar implementation by combining them into a Client superclass. In particular, we ensure that Sellers and Buyers both have personal and banking details set for their accounts. We considered having both Buyer and Seller inherit from RegisteredUser, as Auctioneer does, but decided that the behavior of these two classes needed to be enforced to be similar enough to warrant a superclass. 
\item {\bf BankingDetails: } This class is excluded from the diagram for brevity: we treat it as analogous to the PersonalDetails class, containing the user's account and authorization details, as needed to make transactions. We chose to separate BankingClass from RegisteredUser for loose coupling and high cohesion of the components. This way, RegisteredUser does not need to care if the implementation of BankingDetails changes. This could be particularly useful if we decide we need to validate banking details before a user can register: these changes can be handled within the BankingDetails class, which ensures high cohesion as logically bank validation logic may not necessarily belong in a User class. 
\item {\bf Seller: } This represents a Seller actor. We note the following:
  \begin{itemize}
  \item An instance of the Seller class has some $n$ lots associated with it, the lots it owns. It makes sense to have this stored as an attribute of the seller: it allows for faster retrieval of the lots associated with a seller, rather than the alternative of polling all lots in the system to check which belong to that seller. 
  \item A Seller adding a lot calls the addLot method on itself. This method then calls AuctionHouse.addLot, which creates the lot and adds it to the AuctionHouse's list of lots. This is then returned to the Seller who adds it to the list of their lotsOwned.
  \item We note that the addLot method only needs the LotInformation (described later) and a unique ID associated with the lot, which as per the specification is given to the Seller ahead of time. All other information about a lot can be inferred. However, we note that there is no way given on how an auctioneer is assigned to a lot: we assume this is handled by the auction house staff themselves and do not include how this may be done in the class diagram.
  \item The receiveMessage method only includes support for when the LotUpdateMessage is LotSold, wherein the system informs the Seller that their lot has been sold. We assume other LotUpdateMessages are disregarded, but for the purposes of maintainability, maintain the ability to receive them: for example, it may be desirable in the future for the Seller to be informed when bids are placed on their lot. 
  \end{itemize}
\item {\bf Buyer: } We note the following:
  \begin{itemize}
  \item Like the Seller class, we maintain a list of lots associated with the Buyer. This has the same benefits of not needing to poll all lots to see which lots the Buyer has marked themselves as interested in every time we wish to display these lots.
  \item As per the specification, we have a viewCatalogue() function which returns a List of CatalogueEntries, for the user interface to handle as needed. By separating the retrieval logic from the display logic, we ensure high cohesion. 
  \item The bidOnLot() method only takes an argument of Bid, as the lot it is associated with can be inferred from the bid object.
  \item Both bidOnLot() and markInterestInLot() call the AuctionHouse methods makeBid() and noteInterest methods respectively. This means the Buyer object only communicates with the AuctionHouse singleton, which delegates messages and updates its own status as needed; particularly, it means the Buyer class does not need to be concerned with the interfaces other objects provide and how this may change. 
  \item The receiveMessage() method includes support for all three LotUpdateMessage options:
    \begin{enumerate}
      \item BidPlacedOnLot, so the Buyer is informed when a lot they are interested in has had a bid placed on it;
      \item LotOnAuction, when the lot a Buyer is interested in opens for auction;
      \item LotSold, when the lot a Buyer is interested in is sold. From the Lot object itself we can work out if the Buyer was the winner of this Lot, and change the information presented to the end-user as needed. We originally had three options for this --- LotSold, LotSoldAndWon, LotSoldAndLost --- for the seller, winning buyer, and losing buyers respectively, but this information can be inferred from the Lot object so including separate enums for this is redundant. 
    \end{enumerate}
  \end{itemize}
\item {\bf Auctioneer: } We note the following:
  \begin{itemize}
    \item An Auctioneer has a set of AssignedLots, which is updated by an implied getter / setter not mentioned on the class diagram for brevity. 
    \item The Auctioneer has the ability to open and close a lot, delegating the logic of this to the AuctionHouse class. 
    \item When closing a lot, the auctioneer does not need to specify a winning bid as this can be inferred from the Lot object at the time of the auction closing. 
    \item The Auctioneer is informed when a bid is placed on a lot they are in charge of (BidPlacedOnLot). 
  \end{itemize}
\item {\bf Member of Public: } This user has very restricted privileges, being able only to view catalogue entries. We note that it is an implementation of the Actor interface: this is primarily for maintainability purposes, in case it is decided that members of the public may be allowed to perform more actions --- for example, marking interest in a lot and receiving updates when bids are placed on it.
\item {\bf Lot: } 
\item {\bf LotInformation: } This provides the (mostly) static data associated with the lot. While it would also make sense to store all this information in the Lot object, it is cleaner to abstract it to a separate class. By doing this, we separate the attributes of Lot which often change from those which are set at creation, which is beneficial for brevity and simplifies the interpretation of the Lot class. 
\item {\bf LotStatus: } The Lot Status is, as per Coursework 1, the current state of the Lot. We introduced this state for two reasons:
  \begin{enumerate}
    \item It allows for error-checking: a lot cannot be opened unless it is in a pre-auction state, and closed or bid on unless it is currently on auction.
    \item This provides a convenient method of archiving lots by separating them into those that are still active and those that have been sold and therefore are inactive. This could, for example, allow for catalogue search to show archived lots as well as active lots if this is desired.
  \end{enumerate}
\item {\bf CatalogueEntry: } We assume nothing in particular about the contents of this class. We treat it as the externally-visible representation of a lot, returning it when viewing the catalogue is requested. 
\item {\bf Bid: } It is logical to create a class to represent a Bid on a lot. A Bid contains all the details required to identify it, allowing us to pass around Bid objects rather than a Bid and its associated Lot object. This ensures loose coupling between objects.
\item {\bf BidType: } It is logical to separate the type of a lot. As with the other enums, we originally considered having this as an integer, with code within the class itself to handle the different cases. However, using an enum is conceptually much simpler and more straightforward to understand, keeping with good design principles.
\item {\bf Auctionhouse: } 
\item {\bf Status: } We assume that a given Status can represent either a success or a failure, allowing us to handle possible errors from classes. Otherwise, we assume nothing in particular about this class and do not include it in the class diagram for purposes of brevity. 
\end{itemize}
\section{Dynamic models}
\subsection{UML sequence diagram}
\subsubsection{Close auction}
Due to its size, this is provided as the second-to-last page of the document.

\noindent Some notes on the sequence diagram itself:
\begin{itemize}
  \item We leave out details on how the AuctionHouse class gets such variables as account details. These can easily be inferred from the Buyer and Seller objects, and are left out for brevity. 
  \item We simplify the objects at the top of the sequence diagram, again for brevity. In actuality, we would assume that an external actor would initiate the close method in Auctioneer, and that we would be contacting multiple Buyers.
  \item We leave the process early in two cases:
    \begin{itemize}
      \item If the lot cannot be closed: this occurs if the lot has already been closed or if it was never opened up to auction. In this case we return a failure Status to the Auctioneer.
      \item Similarly, if one transfer cannot be completed when making, we propagate the failure Status down and do not attempt the other transfers, returning a failure Status to the Auctioneer. This is again excluded for brevity.
    \end{itemize}
\end{itemize}

\subsection{Behaviour descriptions}
\subsubsection{Add lot}
In the {\it Add lot} use case, we assume the seller has already created an instance of LotInformation and has a uniqueId for it.\\
Now, they want to create a lot for it and add that to their set of lotsOwned, so they start by invoking the addLot method on themselves.\\
This will involve the AuctionHouse singleton creating new instances of both Lot and CatalogueEntry, and placing them in internal lists. Then, those lot results will be passed back to the seller to add to their own internal list.
\begin{verbatim}
seller -- addLot(LotInformation, uniqueId) ---> seller
  seller ---- createLot(LotInformation, uniqueId) ---> AuctionHouse
    AuctionHouse -- new Lot(LotInformation, uniqueId) ---> Lot
    AuctionHouse <- - - - - newLot - - - - - - - - - - - - Lot
    AuctionHouse -- setLots(newLot) --> AuctionHouse // Add the new lot to the
                                                     // existing "lots" set
                                                     // Use implicit setter
    AuctionHouse -- setCatalogue(newLot) --> AuctionHouse // Create new CatalogueEntry
                                                          // and add to existing
                                                          // internal "catalogue" set
                                                          // Use implicit setter
  seller <- - newLot - - - - - - - - - - - - - - - - - AuctionHouse
                                                          // Return new lot to seller
  seller ---- setLotsOwned(newLot) -> seller // Push the new lot to the internal set
                                             // of lots owned, "lotsOwned"
                                             // Use implicit setter
seller <- - - - - - - - - - - - - - - - - - - - seller
\end{verbatim}
\subsubsection{Note interest}
In the {\it Note interest} use case, we assume the buyer has an instance of lot that they would like to not interest in.\\
This will begin with invoking markInterestInLot on themselves.
\begin{verbatim}
buyer -- markInterestInLot(lot) ---> buyer
  buyer ---- noteInterest(lot, buyer) ---> AuctionHouse
    AuctionHouse ---- setInterestedBuyers(buyer) ---> lot
                                            // Add the buyer to the lot's
                                            // internal "interestedBuyers" set
                                            // Use implicit setter
    AuctionHouse <- - - - - - - - - - - - - - - - - - lot
  buyer <- - - - - - - - - - - - - - - - - AuctionHouse
  buyer ---- setInterestedLots(lot) ----> buyer 
                                            // Add the lot to the buyer's
                                            // internal "interestedLots" set
                                            // Use implicit setter
buyer <- - - - - - - - - - - - - - - buyer
\end{verbatim}
\subsubsection{Make bid}
In the {\it Make bid} use case, we assume the buyer has already created an instance of Bid. The instance of Bid they have created has them as bid.buyer, and the lot they are bidding on as bid.lot. Its value and its type are also already defined.\\
Now, they want to send that bid out to the lot and have it registered in the system.\\
This will begin with invoking bidOnLot on themselves. As the message burst propagates, the AuctionHouse will send out relevant changes to the lot in question, and notify other participants.
\begin{verbatim}
buyer --- bidOnLot(bid) ---> buyer
  buyer ---- makeBid(bid) -----> AuctionHouse
    AuctionHouse --- getLot -------> bid // implicit getter
    AuctionHouse <- - activeLot <- - bid // return lot which bid refers to
    AuctionHouse --- receiveBid(bid) ---> activeLot
    AuctionHouse <- - bidSuccessful - - - activeLot 
    // If bid is insufficient, simply skip the ensuing loop
ALT Bid is insufficient // simply return to buyer and skip the remaining operations
  buyer <- - - bidSuccessful - - AuctionHouse
buyer <- - bidSuccessful - - buyer
ENDALT
ALT Bid is sufficient // continue to notify all relevant actors of new bid
    AuctionHouse --- receiveMessage(BidPlacedOnLot, bid.lot) 
                 --> bid.lot.information.auctioneer 
                // Use implicit getters on current bid to get auctioneer that
                // is currently in charge
                // Notify auctioneer of new bid
    LOOP [over all bid.lot.interestedBuyers] // Use implicit getters on current
                                             // bid to get all current
                                             // interested buyers
      AuctionHouse --- receiveMessage(BidPlacedOnLot, bid.lot) 
                   --> currentInterestedBuyer
    ENDLOOP
  buyer <- - - bidSuccessful - - AuctionHouse
buyer <- - bidSuccessful - - buyer
ENDALT
\end{verbatim}

\includepdf[pages=1,fitpaper]{sequence_diagram}
\includepdf[pages=1,fitpaper]{class_diagram}

\end{document}

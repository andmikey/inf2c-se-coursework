\documentclass[titlepage, 12pt]{extarticle}
\usepackage[margin=1in]{geometry}
\usepackage{tikz}
\usepackage{tikz-uml}
\usepackage{fancyhdr}
\usepackage{lastpage}
\usepackage{pdfpages}
\usepackage{verbatim}

\lhead{Inf2C: SE Coursework 2}
\rhead{s1703773 \& s1737075}
\cfoot{\thepage~/~\pageref{LastPage}}
\pagestyle{fancy}
\begin{document}
\title{{\bf Inf2C: Software Engineering \\Coursework 2 \vspace{2em}\\ Creating a software design for an auction house system}}
\author{
\begin{tabular}{l  c}
  Michael Andrejczuk & s1703773 \\
  Dylan Joseph Thinnes & s1737075
\end{tabular}
}
\date{November 5, 2018}
\maketitle

\tableofcontents
\newpage

\section{Introduction}
We provide a software design for an auction house system, codenamed {\it Auctionista}. Our design is consistent with the requirements design undertaken in Coursework 1. We provide UML class models for the key actors involved in the system, and sequence diagrams and behavior descriptions for use-cases identified in Coursework 1. For further details, please refer to the specification for Coursework 2.

\section{Static model}
\subsection{UML class model}
Due to its size, this is provided as the last page of the document.

\subsection{High-level description}
\noindent Firstly, an overview on a per-concept basis:
\begin{itemize}
\item {\bf Actor: } The actors in the system --- Seller, Buyer, Auctioneer, MemberOfPublic --- are represented as implementations of the Actor interface. This means that:
  \begin{itemize}
  \item By separating interface and implementation, we follow good design principles.
  \item The Actor class ensures a consistent interface for message exchange using the MessageService singleton.
  \item We can encapsulate the handling of messages of different kinds on a per-object basis, making it easy to maintain and add more message types as needed.
  \end{itemize}
\item {\bf MessagingService: } Messages are sent and received only for objects which implement the Actor interface. In the case of this system, all messages passed around the system will relate to a lot. This means that we only define methods for Actors to receive messages which pertain to a given Lot and some change in status (see point below). However, we can easily create new receiveMessage() functions for Actors which take different parameters. 
\item {\bf LotUpdateMessage: } This class specifies all the possible state-changes a lot may undertake. By providing these states as an enum, we ensure maintainability and modularisation: by adding a new possible state to the enum we update all the classes which receive messages pertaining to lots.
\item {\bf RegisteredUser: } This represents superclass the actors who have an account on the system, allowing us to separate them from members of the public, who we assume to not have accounts. Using inheritance, we ensure consistent behavior for these users. Currently, this is only ensuring that all have a username: however, this is easily extensible to also include, for example, passwords and two-factor-authentication.
\item {\bf Client: } A subclass of RegisteredUser, this represents Sellers and Buyers. These two actors use the system in a similar way, so we ensure a similar implementation by combining them into a Client superclass. In particular, we ensure that Sellers and Buyers both have personal and banking details set for their accounts. We considered having both Buyer and Seller inherit from RegisteredUser, as Auctioneer does, but decided that the behavior of these two classes needed to be enforced to be similar enough to warrant a superclass. 
\item {\bf BankingDetails: } This class is excluded from the diagram for brevity: we treat it as analogous to the PersonalDetails class, containing the user's account and authorization details, as needed to make transactions. 
\end{itemize}
\section{Dynamic models}
\subsection{UML sequence diagram}
\subsubsection{Close auction}
Due to its size, this is provided as the second-to-last page of the document.

\noindent Some notes on the sequence diagram itself:
\begin{itemize}
  \item 
\end{itemize}

\subsection{Behaviour descriptions}
\subsubsection{Add lot}
In the {\it Add lot} use case, we assume the seller has already created an instance of LotInformation and has a uniqueId for it.\\
Now, they want to create a lot for it and add that to their set of lotsOwned, so they start by invoking the addLot method on themselves.\\
This will involve the AuctionHouse singleton creating new instances of both Lot and CatalogueEntry, and placing them in internal lists. Then, those lot results will be passed back to the seller to add to their own internal list.
\begin{verbatim}
seller -- addLot(LotInformation, uniqueId) ---> seller
  seller ---- createLot(LotInformation, uniqueId) ---> AuctionHouse
    AuctionHouse -- new Lot(LotInformation, uniqueId) ---> Lot
    AuctionHouse <- - - - - newLot - - - - - - - - - - - - Lot
    AuctionHouse -- setLots(newLot) --> AuctionHouse // Add the new lot to the
                                                     // existing "lots" set
                                                     // Use implicit setter
    AuctionHouse -- setCatalogue(newLot) --> AuctionHouse // Create new CatalogueEntry
                                                          // and add to existing
                                                          // internal "catalogue" set
                                                          // Use implicit setter
  seller <- - newLot - - - - - - - - - - - - - - - - - AuctionHouse
                                                          // Return new lot to seller
  seller ---- setLotsOwned(newLot) -> seller // Push the new lot to the internal set
                                             // of lots owned, "lotsOwned"
                                             // Use implicit setter
seller <- - - - - - - - - - - - - - - - - - - - seller
\end{verbatim}
\subsubsection{Note interest}
In the {\it Note interest} use case, we assume the buyer has an instance of lot that they would like to not interest in.\\
This will begin with invoking markInterestInLot on themselves.
\begin{verbatim}
buyer -- markInterestInLot(lot) ---> buyer
  buyer ---- noteInterest(lot, buyer) ---> AuctionHouse
    AuctionHouse ---- setInterestedBuyers(buyer) ---> lot
                                            // Add the buyer to the lot's
                                            // internal "interestedBuyers" set
                                            // Use implicit setter
    AuctionHouse <- - - - - - - - - - - - - - - - - - lot
  buyer <- - - - - - - - - - - - - - - - - AuctionHouse
  buyer ---- setInterestedLots(lot) ----> buyer 
                                            // Add the lot to the buyer's
                                            // internal "interestedLots" set
                                            // Use implicit setter
buyer <- - - - - - - - - - - - - - - buyer
\end{verbatim}
\subsubsection{Make bid}
In the {\it Make bid} use case, we assume the buyer has already created an instance of Bid. The instance of Bid they have created has them as bid.buyer, and the lot they are bidding on as bid.lot. Its value and its type are also already defined.\\
Now, they want to send that bid out to the lot and have it registered in the system.\\
This will begin with invoking bidOnLot on themselves. As the message burst propagates, the AuctionHouse will send out relevant changes to the lot in question, and notify other participants.
\begin{verbatim}
buyer --- bidOnLot(bid) ---> buyer
  buyer ---- makeBid(bid) -----> AuctionHouse
    AuctionHouse --- getLot -------> bid // implicit getter
    AuctionHouse <- - activeLot <- - bid // return lot which bid refers to
    AuctionHouse --- receiveBid(bid) ---> activeLot
    AuctionHouse <- - bidSuccessful - - - activeLot 
    // If bid is insufficient, simply skip the ensuing loop
ALT Bid is insufficient // simply return to buyer and skip the remaining operations
  buyer <- - - bidSuccessful - - AuctionHouse
buyer <- - bidSuccessful - - buyer
ENDALT
ALT Bid is sufficient // continue to notify all relevant actors of new bid
    AuctionHouse --- receiveMessage(BidPlacedOnLot, bid.lot) 
                 --> bid.lot.information.auctioneer 
                // Use implicit getters on current bid to get auctioneer that
                // is currently in charge
                // Notify auctioneer of new bid
    LOOP [over all bid.lot.interestedBuyers] // Use implicit getters on current
                                             // bid to get all current
                                             // interested buyers
      AuctionHouse --- receiveMessage(BidPlacedOnLot, bid.lot) 
                   --> currentInterestedBuyer
    ENDLOOP
  buyer <- - - bidSuccessful - - AuctionHouse
buyer <- - bidSuccessful - - buyer
ENDALT
\end{verbatim}

\includepdf[pages=1,fitpaper]{sequence_diagram}
\includepdf[pages=1,fitpaper]{class_diagram}

\end{document}
